

A circuit implementation of the staggered model can be found in \cite{MScJaime}. As expected, it resorts to the joint tessellation dynamics, in contrast to a coined quantum walk that uses a coin as a shift operator. For the example discussed above, it yields
\ctikzfig{sqw-circuit}
where $R_x(\theta) = e^{\frac{-i\theta X}{2}}$ and the Inc(rement) and Dec(rement) circuits have the usual implementation through  generalised Toffoli gates.
An  implementation for a 3 qubit staggered quantum walk, starting at $\ket{4}$ and taking $\theta = \frac{\pi}{3}$, which maximizes propagation, is represented in a ZX diagram as

\begin{align*}
    \tikzfig{1-step-sqw}
\end{align*}
~\\

\noindent
which takes advantage of the ZH-calculus H-box notation for representing Toffoli gates in a concise way. Expanding Tofolli into its basic gates leads to 

\begin{align*}
    \tikzfig{1-step-expanded}
\end{align*}
~\\

Some simple optimisations can be considered to deal with the CNOT gate at the end of the expansion of the Toffoli gate and the CNOT belonging to the increment operator, and similarly, but for a X-spider between the CNOT targets,  in the decrement operator. Moreover, one may cancel the two X-spiders with phase $\pi$ in the first qubit between the increment and decrement layers, and the two consecutive Hadamard gates in the last qubit. The result is 
the following ZX-diagram,

\begin{align*}
    \tikzfig{1-step-trivial}
    \label{fig:trivial}
\end{align*}
~\\

These optimisations  slightly reduced the number of Clifford gates in the diagram. More advanced techniques, described in \cite{t-count-opt} and  directly implemented in \textit{PyZX} \cite{pyzx} as the \texttt{full\_reduce} method, may reduce the circuit T-count  in about $50\%$ \cite{t-count-opt}. Although this is not the case for our small example, when we start applying such simplifications to staggered models with  larger amounts of steps the T-count reduction can reach  approximately $60$-$70\%$.
Back to the example, this simplification yields 

\begin{align*}
    \tikzfig{graph-like}
\end{align*}
~\\

This diagram no longer resembles a circuit, making a comparison with the original one difficult. The circuit extracted \cite{extraction-p-hard} by \textit{PyZX} has 
more gates than the one obtained from the simple optimisations mentioned above,  although the T-count is indeed smaller.
  
%\begin{align}
%    \tikzfig{fullreduce-simple}
%\end{align}

In fact, the \texttt{full\_reduce} method introduces several additional Hadamard gates  to make all the edges Hadamard-edges, and color-changes all the X-spiders. 
The subsequent  circuit extraction  'preserves' the nature of the \textit{graph-like} ZX-diagram. As such there are a lot of Hadamard gates and Z-spiders that could cancel-out or change color. Following the extraction with a small set of simplifications, basically resorting to fusion rewrite rules followed by a color-change, we get a much smaller circuit. 

This fully-simplified circuit now surpasses the original circuit in both the total amount of gates and T-count, but not the one obtained with the simple optimisations above. Although the reduction of both these metrics are not that significant in this example, when applying the same techniques in models with a greater number of steps  reductions in the number of gates and T-count become quite clear. The following tables show, respectively, the total number of  gates and the T-count value induced by the different optimisation procedures used.

\begin{table}[H]
\centering
\begin{tabular}{|l|cccc|}
\hline
                               & \multicolumn{4}{l|}{Number of steps in the staggered quantum walk:}                                   \\ \hline
Optimizations used:          & \multicolumn{1}{c|}{1}  & \multicolumn{1}{c|}{2}  & \multicolumn{1}{c|}{4}   & 8   \\ \hline
None            & \multicolumn{1}{c|}{39} & \multicolumn{1}{c|}{77} & \multicolumn{1}{c|}{153} & 305 \\ \hline
Simple     & \multicolumn{1}{c|}{31} & \multicolumn{1}{c|}{59} & \multicolumn{1}{c|}{115} & 227 \\ \hline
Full-reduce + fusion/id/to\_rg & \multicolumn{1}{c|}{37} & \multicolumn{1}{c|}{47} & \multicolumn{1}{c|}{72}  & 118 \\ \hline
\end{tabular}
%\caption{Number of total gates in the circuit in relation to the simplification routines used}
%\label{tab:total-gates}
\end{table}
\vspace{-1cm}
\begin{table}[H]
\centering
\begin{tabular}{|l|cccc|}
\hline
                               & \multicolumn{4}{l|}{Number of steps in the staggered quantum walk:}                                  \\ \hline
Optimizations used:          & \multicolumn{1}{c|}{1}  & \multicolumn{1}{c|}{2}  & \multicolumn{1}{c|}{4}  & 8   \\ \hline
None            & \multicolumn{1}{c|}{16} & \multicolumn{1}{c|}{32} & \multicolumn{1}{c|}{64} & 128 \\ \hline
Simple       & \multicolumn{1}{c|}{16} & \multicolumn{1}{c|}{32} & \multicolumn{1}{c|}{64} & 128 \\ \hline
Full-reduce + fusion/id/to\_rg & \multicolumn{1}{c|}{10} & \multicolumn{1}{c|}{16} & \multicolumn{1}{c|}{28} & 52  \\ \hline
\end{tabular}
\end{table}

%The Staggered Quantum Walk is already suitable for the current NISQ landscape \cite{MScJaime}, with these simplification routines we can minimize even further noise induced errors and decoherence.










